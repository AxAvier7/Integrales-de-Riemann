\documentclass{article}
\usepackage{amsmath}
\usepackage{amsfonts}
\usepackage{amssymb}
\usepackage{amsthm}
\newtheorem{teorema}{Teorema}
\newtheorem{definición}{Definición}
\newtheorem{ejemplo}{Ejemplo}
\renewcommand{\abstractname}{Resumen}
\renewcommand{\proofname}{Demostración}

\title{Condición Necesaria y Suficiente de Integrabilidad}
\author{Brioso Jurado, Adrián Xavier | Reyes García, Elisabeth}
\date{\today}

\begin{document}
\maketitle

\begin{abstract}
En el presente documento se explica la condición necesaria y suficiente de integrabilidad según Riemann.
\end{abstract}

\section{Condición Necesaria y Suficiente de Integrabilidad según Riemann}

\subsection{Definición}
Una función \(f(x)\) se dice \textit{integrable según Riemann} en un intervalo \([a,b]\) si existe el límite de las sumas integrales:
\[
\int_{a}^{b} f(x) \, dx = \lim_{\lambda \to 0} \sigma(P, \xi_i),
\]
donde \( \lambda = \max \Delta x_i \) y \( \sigma(P, \xi_i) = \sum_{i=1}^{n} f(\xi_i) \Delta x_i \).\\

De igual forma, se puede definir la integrabilidad mediante las \textbf{sumas de Darboux}:
\begin{itemize}
    \item \textbf{Suma superior de Darboux:} \( S(P, f) = \sum_{i=1}^{n} M_i \Delta x_i \), donde \( M_i = \sup \{f(x) \mid x \in [x_{i-1}, x_i]\} \).
    \item \textbf{Suma inferior de Darboux:} \( s(P, f) = \sum_{i=1}^{n} m_i \Delta x_i \), donde \( m_i = \inf \{f(x) \mid x \in [x_{i-1}, x_i]\} \).
\end{itemize}
Se puede decir entonces que una función \textit{f} es Riemann integrable si:
\[
f \in \mathbb{R}[a,b] \iff \forall \epsilon>0: S(f,P) - s(f,P)< \epsilon \iff \underline{I} = \overline{I}
\]\\

Por tanto,
\[
\int_a^b f = \lim_{\left\{ P \right\}} S(f,P) = \lim_{\left\{ P \right\}} s(f,P)
\]

\subsection{Condiciones}
\begin{enumerate}
    \item \textbf{Condición necesaria:} \( f(x) \) debe ser \textbf{acotada} en \([a, b]\).
        \begin{proof}
            Sea \( f \) continua en \([a, b]\). Por el \textbf{Teorema de Heine-Cantor}, \( f \) es uniformemente continua en \([a, b]\). Dado \( \epsilon > 0 \), existe \( \delta > 0 \) tal que:
            \[
            |x - y| < \delta \implies |f(x) - f(y)| < \epsilon.
            \]
            Consideremos una partición \( P \) de \([a, b]\) con \( \lambda = \max \Delta x_i < \delta \). Para las sumas de Darboux:
            \[
            S(f, P) - s(f, P) = \sum_{i=1}^n (M_i - m_i) \Delta x_i,
            \]
            donde \( M_i = \sup f \) y \( m_i = \inf f \) en \([x_{i-1}, x_i]\). Por continuidad uniforme:
            \[
            M_i - m_i < \frac{\epsilon}{b-a} \quad \forall i.
            \]
            Luego:
            \[
            S(f, P) - s(f, P) < \epsilon.
            \]
            Como \( \epsilon \) es arbitrario, \( \lim_{\lambda \to 0} (S - s) = 0 \). Por tanto, \( f \) es integrable.
            \end{proof}
    
    \item \textbf{Condición suficiente:} \( f(x) \) es integrable si es acotada y su conjunto de discontinuidades es finito.
    \begin{itemize}
        \item \textbf{Condición 1:} Las funciones que sean continuas en \([a, b]\) son integrables en ese intervalo.
            \begin{proof}
                Sea \( f \) continua en \([a, b]\). Por el \textbf{Teorema de Heine-Cantor}, \( f \) es uniformemente continua. Dado \( \epsilon > 0 \), existe \( \delta > 0 \) tal que:
                \[
                |x - y| < \delta \implies |f(x) - f(y)| < \frac{\epsilon}{b-a}
                \]
                Para una partición \( P \) con \( \lambda < \delta \), en cada subintervalo \([x_{i-1}, x_i]\):
                \[
                M_i - m_i < \frac{\epsilon}{b-a}
                \]
                Luego:
                \[
                S(P,f) - s(P,f) = \sum_{i=1}^n (M_i - m_i)\Delta x_i < \epsilon
                \]
                Como \( \epsilon \) es arbitrario, \( f \) es integrable.
            \end{proof}
            
        \item \textbf{Condición 2:} Toda función monótona en \([a, b]\) es integrable.
            \begin{proof}
                Sea \( f \) monótona creciente en \([a, b]\) (análogo para funciones decrecientes). Por ser monótona, \( f \) está acotada (\( f(a) \leq f(x) \leq f(b) \)). Dado \( \epsilon > 0 \), consideremos una partición \( P \) de \([a, b]\) en \( n \) subintervalos de igual longitud \( \Delta x = \frac{b - a}{n} \). 

                Para las sumas de Darboux:
                \[
                S(P,f) - s(P,f) = \sum_{i=1}^n \left(f(x_i) - f(x_{i-1})\right)\Delta x,
                \]
                donde \( M_i = f(x_i) \) y \( m_i = f(x_{i-1}) \) por la monotonía. Al simplificar:
                \[
                \sum_{i=1}^n \left(f(x_i) - f(x_{i-1})\right) = f(b) - f(a),
                \]
                se obtiene:
                \[
                S(P,f) - s(P,f) = \Delta x \cdot \left(f(b) - f(a)\right) = \frac{(b - a)(f(b) - f(a))}{n}.
                \]
                
                Despejando \( n \):
                \[
                \frac{(b - a)(f(b) - f(a))}{n} < \epsilon \quad \implies \quad n > \frac{(b - a)(f(b) - f(a))}{\epsilon}.
                \]
                Con esta \( n \), se cumple \( S(P,f) - s(P,f) < \epsilon \). Por tanto, \( f \) es integrable.
            \end{proof}
            
        \item \textbf{Condición 3:} Las funciones con discontinuidades en un conjunto finito de puntos son integrables. Al modificar la función en  puntos aislados el valor de la integral no se ve afectado.
            \begin{proof}
                Sea \( f: [a,b] \to \mathbb{R} \) acotada con discontinuidades solo en \( \{c_1, c_2, ..., c_n\} \). Demostraremos que \( f \in \mathbb{R}[a,b] \).

                Como \( f \) es acotada, existe \( M > 0 \) tal que \( |f(x)| \leq M \) para todo \( x \in [a,b] \).

                Dado \( \epsilon > 0 \), creamos una vecindad alrededor de cada discontinuidad \( c_k \):
                \[
                I_k = \left(c_k - \frac{\epsilon}{8nM}, c_k + \frac{\epsilon}{8nM}\right) \cap [a,b], \quad k = 1, \ldots, n.
                \]
                La unión de estos intervalos tiene longitud total \( \leq n \cdot \frac{\epsilon}{4nM} = \frac{\epsilon}{4M} \).

                Fuera de \( \bigcup I_k \) (la unión de todas las discontinuidades), \( f \) es continua. Por el Teorema de Heine-Cantor, es uniformemente continua en el conjunto cerrado \( [a,b] \setminus \bigcup I_k \). Existe \( \delta > 0 \) tal que:
                \[
                |x - y| < \delta \implies |f(x) - f(y)| < \frac{\epsilon}{2(b - a)}.
                \]

                Tomemos una partición \( P \) que incluya todos los extremos de los \( I_k \) y tenga norma \( \lambda < \min\left\{\delta, \frac{\epsilon}{8nM}\right\} \).

                Separamos la suma en dos partes:
                \begin{align*}
                    S(P,f) - s(P,f) &= \sum_{\text{subintervalos en } \bigcup I_k} (M_i - m_i)\Delta x_i + \sum_{\text{resto}} (M_i - m_i)\Delta x_i \\
                    &\leq \sum_{\bigcup I_k} 2M \Delta x_i + \sum_{\text{resto}} \frac{\epsilon}{2(b - a)} \Delta x_i \\
                    &\leq 2M \cdot \frac{\epsilon}{4M} + \frac{\epsilon}{2(b - a)} \cdot (b - a) \\
                    &= \frac{\epsilon}{2} + \frac{\epsilon}{2} = \epsilon.
                \end{align*}

                Si \( g \) es distinta de \( f \) solo en \( \{c_1, ..., c_m\} \), entonces \( |f - g| \neq 0 \) solo en esos puntos. Como los puntos tienen medida cero:
                \[
                \int_a^b |f(x) - g(x)| dx = 0 \implies \int_a^b f(x) dx = \int_a^b g(x) dx.
                \]
            \end{proof}

    \end{itemize}
\end{enumerate}

\subsection*{Ejemplos de Funciones Integrables según las Condiciones Suficientes}

\begin{enumerate}
    \item \textbf{Condición 1: Funciones continuas en \([a, b]\).}
        \begin{ejemplo}
            Sea \( f(x) = e^x \) en \([0, 1]\). Esta función es continua en todo su dominio. Por el \textbf{Teorema de Heine-Cantor}, al ser continua en un intervalo cerrado, es uniformemente continua y por tanto integrable y su integral es:
            \[
            \int_0^1 e^x \, dx = e - 1.
            \]
        \end{ejemplo}
        
    \item \textbf{Condición 2: Funciones monótonas en \([a, b]\).}
        \begin{ejemplo}
            Sea \( f: [0, 3] \to \mathbb{R} \) la función escalonada:
            \[
            f(x) = 
            \begin{cases}
                0, & 0 \leq x < 1 \\
                2, & 1 \leq x < 2 \\
                5, & 2 \leq x \leq 3
            \end{cases}
            \]
            \( f \) es monótona creciente y tiene discontinuidades en \( x = 1 \) y \( x = 2 \). Aunque presenta saltos finitos es integrable:
            \[
            \int_0^3 f(x) \, dx = \int_0^1 f(x) \, dx + \int_1^2 f(x) \, dx + \int_2^3 f(x) \, dx =
            0(1-0) + 2(2-1) + 5(3-2) = 7.
            \]
        \end{ejemplo}

    \item \textbf{Condición 3: Funciones con discontinuidades finitas.}
        \begin{ejemplo}
            Sea \( f: [0, 2] \to \mathbb{R} \) definida por:
            \[
            f(x) = 
            \begin{cases}
                0, & x \neq 1 \\
                5, & x = 1
            \end{cases}
            \]
            \( f \) es discontinua únicamente en \( x = 1 \). Como el conjunto de discontinuidades es finito, \( f \) es integrable y:
            \[
            \int_0^2 f(x) \, dx = 0.
            \]
            \begin{proof}
                Las modificaciones en puntos aislados no afectan la integral. La función no es nula en \( x = 1 \) y por tanto tiene medida cero.
            \end{proof}
        \end{ejemplo}

\end{enumerate}

\subsection*{Contraejemplo: Función No Integrable}
\begin{ejemplo}
    La \textbf{función de Dirichlet} en \([0, 1]\):
    \[
    D(x) = 
    \begin{cases}
        1, & x \in \mathbb{Q} \\
        0, & x \in \mathbb{R} \setminus \mathbb{Q}
    \end{cases}
    \]
    No es integrable según Riemann, ya que es discontinua en todo el intervalo (\( \mathbb{Q} \cap [0,1] \) es denso y no tiene medida cero). Para cualquier partición \( P \):
    \[
    S(P,D) = 1 \quad \text{y} \quad s(P,D) = 0 \implies S - s = 1 \not< \epsilon.
    \]
\end{ejemplo}

\subsection*{Conclusiones}
La \textbf{integrabilidad según Riemann} requiere: 
\[
\boxed{
    \parbox{10cm}{
            \centering
            \( f(x) \) es integrable en \([a, b]\) \\[0.5em]
            \(\iff\) \\[0.5em]
            \( f(x) \) es acotada en [a,b] y se cumple una de las siguientes condiciones:
            \begin{itemize}
                \item \(f\) es continua en \([a,b]\).
                \item \(f\) es monótona en \([a,b]\).
                \item \(f\) es presenta discontinuidades finitas.
            \end{itemize}
    }
}
\]

\end{document}