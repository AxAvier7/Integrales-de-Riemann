\documentclass{article}
\usepackage{amsmath}
\usepackage{amsfonts}
\renewcommand{\abstractname}{Resumen}

\title{Condición Necesaria y Suficiente de Integrabilidad}
\author{Brioso Jurado, Adrián Xavier | Reyes García, Elisabeth}
\date{\today}

\begin{document}
\maketitle

\begin{abstract}
En el presente documento se explica la condición necesaria y suficiente de integrabilidad según Riemann.
\end{abstract}

\section{Condición Necesaria y Suficiente de Integrabilidad según Riemann}

\subsection{Definición}
Una función \(f(x)\) se dice \textit{integrable según Riemann} en un intervalo \([a,b]\) si existe el límite de las sumas integrales:
\[
\int_{a}^{b} f(x) \, dx = \lim_{\lambda \to 0} \sigma(P, \xi_i),
\]
donde \( \lambda = \max \Delta x_i \) y \( \sigma(P, \xi_i) = \sum_{i=1}^{n} f(\xi_i) \Delta x_i \).

\subsection{Condiciones}
\begin{enumerate}
    \item \textbf{Condición necesaria:} \( f(x) \) debe ser \textbf{acotada} en \([a, b]\).
    \begin{itemize}
            \item Si \( f(x) \) no está acotada, las sumas integrales pueden divergir (ver demostración con funciones no acotadas).
    \end{itemize}

    \item \textbf{Condición suficiente:} \( f(x) \) es integrable si es acotada y su conjunto de discontinuidades es finito.
    \begin{itemize}
        \item \textbf{Ejemplo 1:} Las funciones que sean continuas en \([a, b]\) son integrables en ese intervalo.

        \item \textbf{Ejemplo 2:} Las funciones con discontinuidades en un conjunto finito de puntos son integrables. Al modificar la función en  puntos aislados el valor de la integral no se ve afectado.

        \item \textbf{Contraejemplo:} La función de Dirichlet no es integrable ya que es discontinua en todo el intervalo \([a, b]\).

    \end{itemize}
\end{enumerate}

\subsection*{Demostraciones y Ejemplos}
\begin{itemize}
    \item \textbf{Función constante:} \( f(x) = c \) es integrable en \([a, b]\),
    y \[
    \int_{a}^{b} c \, dx = c(b-a).
    \]

    \item \textbf{Función no nula en un punto:} Si \( f(x) = 0 \) excepto en un punto, entonces \( \int_{a}^{b} f(x) dx = 0 \). Esto se puede ampliar a un número finito de puntos.  

    \item \textbf{Función de Dirichlet:}
    \[
    D(x) =
    \begin{cases}
        0, & x \in \mathbb{R} \setminus \mathbb{Q} \\
        1, & x \in \mathbb{Q}
    \end{cases}
    \]
    No es integrable en ningún \( [a, b] \) ya que las sumas integrales alternan entre \(0\) y \( b -a \).
\end{itemize}

\subsection*{Conclusiones}
La \textbf{integrabilidad según Riemann} requiere: 
\[
\boxed{
    \parbox{8cm}{
            \centering
            \( f(x) \) es integrable en \([a, b]\) \\[0.5em]
            \(\iff\) \\[0.5em]
            \( f(x) \) es acotada y presenta una cantidad finita de discontinuidades.
    }
}
\]

\end{document}